\documentclass[slides]{pgnotes}

\title{Lambda function URLs}

\begin{document}

\maketitle

\section{Purpose}\label{purpose}

\begin{itemize}
\item Lambda function URLs are a simple way to provide HTTPS endpoints for a Lambda function.
\item The incoming HTTPS \textit{request} is converted to the Lambda \textbf{event} object.
\item The \textbf{return} object is converted to the HTTPS \textit{response}.
\end{itemize}

\section{Event object}\label{event-object}

The event object contains the HTTP request info as a Python dictionary.

This has a particular defined structure when called from a Lambda function URL.

\subsection{Query string parameters}\label{query-string-parameters}

Query string parameters often supply input to web APIs:

\begin{verbatim}
https://api.name/function?name=Peadar&town=Dundalk
\end{verbatim}

These can be accessed by name through
\begin{minted}{Python}
event['queryStringParameters']
\end{minted}
  
\section{Return object}\label{return-object}

The return object must obey a specific structure. As a minimum, the
Python dictionary must have:

\begin{verbatim}
{
    'statusCode': 200,
    'body': body_variable
}
\end{verbatim}

Note:
\begin{itemize}
\item The statusCode normally is 200 indicating a successful request.
\item The body is \emph{text} returned to the browser.
\end{itemize}

\subsection{Body}\label{body}

The body must be as plain text.

Python dictionary structures should be converted using:

\begin{verbatim}
body = json.dumps(my_dict)
\end{verbatim}

This can be confusing since the outer event and returned objects are converted from/to JSON transparently!

\end{document}

